\documentclass{article}
\usepackage[margin=1in]{geometry}

\usepackage[colorlinks = true,
            linkcolor = blue,
            urlcolor  = blue,
            citecolor = blue,
            anchorcolor = blue]{hyperref}

\newcommand{\MYhref}[3][blue]{\href{#2}{\color{#1}{#3}}}%

\title{Using VsCode on Grace (with MacOS)}
\author{Rami Pellumbi}
\date{\today}

\begin{document}

\maketitle

\section{SSH Setup}

To set SSH up on VsCode, first install the \MYhref{https://marketplace.visualstudio.com/items?itemName=ms-vscode-remote.remote-ssh}{Remote - SSH exension from Microsoft} 
via the Extensions: Marketplace. Next, you will have to generate your key pair. The instructions can be found 
on the \MYhref{https://docs.ycrc.yale.edu/clusters-at-yale/access/ssh/}{YCRC SSH docs}. For completeness, I state how I 
set up my configuration on my MacOS machine. The documentation is copied from the YCRC SSH documentation. Open your terminal and type:
\begin{verbatim}
    ssh-keygen    
\end{verbatim}
Your terminal should respond:
\begin{verbatim}
Generating public/private rsa key pair.
Enter file in which to save the key (/home/yourusername/.ssh/id_rsa):
\end{verbatim}
Press Enter to accept the default value.\footnote{If you have already created an SSH key, chances are the \texttt{id\_rsa} name is already in use and you will want to rename it. For example, 
I had my GitHub key using my \texttt{id\_rsa} ssh-key so I named mine \texttt{id\_rsa\_grace}.}
Your terminal should respond:
\begin{verbatim}
Enter passphrase (empty for no passphrase):
\end{verbatim}
Enter a secure password. The response will be:
\begin{verbatim}
Enter same passphrase again:
\end{verbatim}
Next, upload your public SSH key on the cluster. Run the following command in terminal to copy the contents of 
your public SSH key\footnote{Replace \texttt{id\_rsa.pub} with the name of your ssh key if you did not use the default name.}
\begin{verbatim}
    cat ~/.ssh/id_rsa.pub
\end{verbatim}
Next, head to the \MYhref{https://sshkeys.ycrc.yale.edu/}{SSH key uploader} and paste the contents.

\section{Connecting to Grace}

To make connecting to Grace more simply, we will modify our ssh configuration file. 
Open your terminal and head to your ssh-directory.
\begin{verbatim}
    cd ~/.ssh
\end{verbatim}
Open your config file (in the editor of your choice) located at \texttt{~/.ssh/config}.\footnote{Create the file if it does not exist.}
A simple way to do this is to run 
\begin{verbatim}
    vim config
\end{verbatim}
from the current directory. Paste the following contents into the file\footnote{It is VERY important you point to the 
right \texttt{IdentityFile} and replace the \texttt{User} with your \texttt{cpsc424\_YOURNETID}.}
\begin{verbatim}
# ENSURE THIS POINTS TO YOUR SSH KEY THAT YOU UPLOADED TO GRACE
IdentityFile ~/.ssh/id_rsa_grace

# Uncomment the ForwardX11 options line to enable X11 Forwarding by default 
# (no -Y necessary)
# On a Mac you still need xquartz installed

Host *.ycrc.yale.edu grace
    User cpsc424_rp862          # REPLACE THIS WITH YOUR YOUR NET ID
    ForwardX11 yes
    # To re-use your connections with multi-factor authentication
    # Uncomment the two lines below
    ControlMaster auto
    ControlPath ~/.ssh/tmp/%h_%p_%r

Host grace
    HostName %h.ycrc.yale.edu
\end{verbatim}
Now, in VsCode, hit \texttt{Shift + Command + P} to open the command palette and enter \texttt{Remote-SSH: Connect to Host}. You will 
be presented with the option `grace' in the list of options. Select it and a new window will open up prompting you for the password to your ssh key. 
Your visual studio code now can edit files on Grace.
\end{document}